%%%%%%%%%%%%%%%%%
% This is an sample CV template created using altacv.cls
% (v1.1.5, 1 December 2018) written by LianTze Lim (liantze@gmail.com). Now compiles with pdfLaTeX, XeLaTeX and LuaLaTeX.
%
%% It may be distributed and/or modified under the
%% conditions of the LaTeX Project Public License, either version 1.3
%% of this license or (at your option) any later version.
%% The latest version of this license is in
%%    http://www.latex-project.org/lppl.txt
%% and version 1.3 or later is part of all distributions of LaTeX
%% version 2003/12/01 or later.
%%%%%%%%%%%%%%%%

%% If you need to pass whatever options to xcolor
\PassOptionsToPackage{dvipsnames}{xcolor}

%% If you are using \orcid or academicons
%% icons, make sure you have the academicons
%% option here, and compile with XeLaTeX
%% or LuaLaTeX.
% \documentclass[10pt,a4paper,academicons]{altacv}

%% Use the "normalphoto" option if you want a normal photo instead of cropped to a circle
% \documentclass[10pt,a4paper,normalphoto]{altacv}

\documentclass[10pt,a4paper,ragged2e]{altacv}

\usepackage[hidelinks]{hyperref}


%% AltaCV uses the fontawesome and academicon fonts
%% and packages.
%% See texdoc.net/pkg/fontawecome and http://texdoc.net/pkg/academicons for full list of symbols. You MUST compile with XeLaTeX or LuaLaTeX if you want to use academicons.

% Change the page layout if you need to
\geometry{left=1cm,right=9cm,marginparwidth=6.8cm,marginparsep=1.2cm,top=1.25cm,bottom=1.25cm}

% Change the font if you want to, depending on whether
% you're using pdflatex or xelatex/lualatex
\ifxetexorluatex
  % If using xelatex or lualatex:
  \setmainfont{Lato}
\else
  % If using pdflatex:
  \usepackage[utf8]{inputenc}
  \usepackage[T1]{fontenc}
  \usepackage[default]{lato}
\fi

% Change the colours if you want to
\definecolor{Black}{HTML}{E15F00}
\definecolor{SlateGrey}{HTML}{2E2E2E}
\definecolor{LightGrey}{HTML}{666666}
\colorlet{heading}{SlateGrey}
\colorlet{accent}{Black}
\colorlet{emphasis}{SlateGrey}
\colorlet{body}{LightGrey}

% Change the bullets for itemize and rating marker
% for \cvskill if you want to
\renewcommand{\itemmarker}{{\small\textbullet}}
\renewcommand{\ratingmarker}{\faCircle}

%% sample.bib contains your publications
\addbibresource{sample.bib}

\begin{document}
\name{Andrew Garcia}
\tagline{Ph.D. student with a passion to solve real-world problems through first principles \& statistical approaches}
% \photo{2.8cm}{Globe_High}
\personalinfo{%
  % Not all of these are required!
  % You can add your own with \printinfo{symbol}{detail}
  \email{garcia.gtr@gmail.com}
  \phone{786-332-9418}
  % \mailaddress{560 Constitution Dr. Orlando, FL 32809}
  \location{Gainesville, FL}
  \homepage{\href{https://scholar.google.com/citations?hl=en&user=75udJTAAAAAJ}{Google Scholar} }
  % \twitter{@twitterhandle}
  \github{  \href{https://github.com/andrewrgarcia}{github.com/andrewrgarcia} }

  \linkedin{  \href{https://www.linkedin.com/in/andrewrygarcia/}{/in/andrewrygarcia} }
  %% You MUST add the academicons option to \documentclass, then compile with LuaLaTeX or XeLaTeX, if you want to use \orcid or other academicons commands.
  % \orcid{orcid.org/0000-0000-0000-0000}
}

%% Make the header extend all the way to the right, if you want.
\begin{fullwidth}
\makecvheader
\end{fullwidth}

%% Depending on your tastes, you may want to make fonts of itemize environments slightly smaller
% \AtBeginEnvironment{itemize}{\small}

%% Provide the file name containing the sidebar contents as an optional parameter to \cvsection.
%% You can always just use \marginpar{...} if you do
%% not need to align the top of the contents to any
%% \cvsection title in the "main" bar.
\cvsection[arg-p1sidebar]{Experience}

\cvevent{Graduate Assistant}{University of Florida}{08/2017 -- Ongoing}{Gainesville, FL}
\begin{itemize}
\item Made a kinetic Monte Carlo (kMC) Python script for a crystal layer dissolution process to optimize crystal size through a cycling process
\item Tutored college students on a Python programming based course (COT3502) as a teaching assistant for 1 semester
\end{itemize}

\divider

\cvevent{Associate Engineer}{Xerox}{07/2015--07/2017}{Webster, NY}
\begin{itemize}
\item Provided estimates of spread (standard deviation) through factorial and simple Monte Carlo (sMC) methods for a
system level design which was implemented at the production scale.
% \item Job description 2
\end{itemize}

\divider

\cvevent{Research Assistant}{University of Florida}{12/2013--06/2015}{Gainesville, FL}
\begin{itemize}
% \item Materialized primary goals of research project, making helpful contributions to the passing of a larger NIH R01 funded
% (about \$180,000 for the year 2015) project.
\item  Co-invented a technology highly applicable to the \$1.68 billion dollar market of nerve repair and regeneration.
% \item Prepared plots for a publication using Python and submitted some to an entry to the 2015 Scipy John Hunter plotting
% contest
\end{itemize}



\cvsection{Projects}

\cvevent{ \href{https://doi.org/10.1016/j.colsurfa.2017.05.058/}{Processing-size correlations for the synthesis of magnetic alginate microspheres}}{University of Florida}{2017}{}
\begin{itemize}
\item A model for estimating the size of crosslinked microspheres based on processing conditions was developed from power law fits.
\item Standard error of 3\% was reported for the effect of shear rate with respect to size.
\item Article published in \href{https://doi.org/10.1016/j.colsurfa.2017.05.058/}{\textit{Colloids Surf., A} }

\end{itemize}

\divider

\cvevent{Prediction of a macroscopic property measuring technique through first principles microscopic dynamics}{Xerox}{One week}{}
\begin{itemize}
\item Created an algorithm which was able to extrapolate with high confidence ($R^2$ > 0.98) the outputs of a confidential characterization technique from a single-point stochastic fit.
\item Algorithm was based on the acceptance-rejection sampling principle of Monte Carlo.
\end{itemize}



% \medskip
%
% \cvsection{A Day of My Life}
%
% % Adapted from @Jake's answer from http://tex.stackexchange.com/a/82729/226
% % \wheelchart{outer radius}{inner radius}{
% % comma-separated list of value/text width/color/detail}
% \wheelchart{1.5cm}{0.5cm}{%
%   6/8em/accent!30/{Sleep,\\beautiful sleep},
%   3/8em/accent!40/Hopeful novelist by night,
%   8/8em/accent!60/Daytime job,
%   2/10em/accent/Sports and relaxation,
%   5/6em/accent!20/Spending time with family
% }
%
% \clearpage
% \cvsection[page2sidebar]{Publications}
%
% \nocite{*}
%
% \printbibliography[heading=pubtype,title={\printinfo{\faBook}{Books}},type=book]
%
% \divider
%
% \printbibliography[heading=pubtype,title={\printinfo{\faFileTextO}{Journal Articles}},type=article]
%
% \divider
%
% \printbibliography[heading=pubtype,title={\printinfo{\faGroup}{Conference Proceedings}},type=inproceedings]

%% If the NEXT page doesn't start with a \cvsection but you'd
%% still like to add a sidebar, then use this command on THIS
%% page to add it. The optional argument lets you pull up the
%% sidebar a bit so that it looks aligned with the top of the
%% main column.
% \addnextpagesidebar[-1ex]{page3sidebar}


\end{document}
